\documentclass[a4paper,10pt]{article}
\usepackage[left=1in,right=1in,top=1in,bottom=1in]{geometry}

\begin{document}

\pagestyle{empty}
\begin{center}

Math 2400 \hspace{2cm} \textbf{\LARGE CENTER OF MASS MOBILE} \hspace{2cm} P3
\end{center}
\hrule
\section*{}
We need some more Math Department decorations, so we're going to build a mobile that demonstrates our mastery of the concept of the center of mass of a two dimentional object. This is partially a group effort, since the final product is a single mobile, but each participant will contribute one shape and submit their own write-up of the work.
\\
\\


The basic process will be:\\
\begin{enumerate}
\item Design and build a cool shape. This should be basically two-dimensional, but make sure that your shape is not uniform density.
\item If you need to, put a backing (and fronting?) on your shape to ensure pieces stay together.
\item Find the center of mass of that object. 
\item Test it and make sure it is reasonable. 
\item Clearly mark the coordinates on the object itself as well as the total mass.
\item Attach a string to the object at the center of mass so that it will hang from the string level with the floor.
\item Hang your object from the mobile in a way that keeps the entire mobile balanced.

\end{enumerate}

\section*{}
I have some shapes that you can use as pieces, tape, scissors, card-stock, and other helpful things available in my office for building. Stop by to pick up some materials, or you can absolutely build from anything you've already got. I'll maintain the mobile in my office, and have tools and materials available for hanging. You may come by anytime to add your shape to the mobile; you don't need an appointment, but you are welcome to check with me first to make sure I'm around. You should allow a few minutes to do the necessary computations and adjustments to keep it Balanced. \\

The write-up you submit on Blackboard should include
\begin{itemize}
\item A diagram of the complete shape and all the specifications.
\item The steps to compute the center of mass.
\item An explanation of the computation and derivation of any new formulas, and
\item A summery of any collaboration and computations needed to hang the final object and keep the whole system balanced.Anean on af coloratation and piration feede forts und object and keep the whole system balanced.
\end{itemize}

\section*{}
The next page has a grid of 1 centimeter squares. That can be helpful for tracing out your shapes, but you are not required to use it.
\end{document}
